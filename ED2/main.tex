\documentclass{homework}

\title{Estudo Dirigido 2}
\author{Carlos Henrique Bravo Serrado\\ 119136241}

\begin{document}

\maketitle

\exercise*
Começando por A. Encontrar o polinômio característico:
\[det \left( \frac{1}{10}
\begin{bmatrix}
-7 & 9\\
9 & 17
\end{bmatrix} - 
\begin{bmatrix}
t & 0\\
0 & t
\end{bmatrix}\right) = \left(-\frac{7}{10}-t\right)\left(\frac{17}{10}-t\right)-\frac{81}{100}
= t^2 - t - 2 = (t+1)(t-2)\]
Então, os autovalores de A são $2$ e $-1$. Para os autovetores associados a 2:
\[\frac{1}{10}\begin{bmatrix}
-27x_1 + 9x_2\\
9x_1 - 3x_2
\end{bmatrix} = \begin{bmatrix}
0 \\
0
\end{bmatrix}\]
\[x_2 = 3x_1\]
Então, os resultados são da forma $(x_1,3x_1)$, ou, $x_1(1,3)$, com $x_1 \neq 0$. Como a matriz A é simétrica ($a_{12} = a_{21}$), podemos usar do Teorema Espectral que os autovetores são ortogonais, então o autovetor associado a -1 é $\lambda(-3,1)$, com $\lambda \neq 0$. Os autovetores da matriz A são:
\[V_2 = \langle(1,3)\rangle\]
\[V_{-1} = \langle(-3,1)\rangle\]

Para B, primeiro encontrar o polinômio característico:
\[det \left( \frac{1}{25}
\begin{bmatrix}
43 & -24\\
-24 & 67
\end{bmatrix} - 
\begin{bmatrix}
t & 0\\
0 & t
\end{bmatrix}\right) = \left(\frac{43}{25}-t\right)\left(\frac{57}{25}-t\right)-\frac{576}{625}
= t^2 - 4t + 3 = (t-3)(t-1)\]
Então B tem os autovalores $3$ e $1$. Para os autovetores associados a 1:
\[\frac{1}{25}\begin{bmatrix}
18x_1 -24x_2\\
-24x_1 + 32x_2
\end{bmatrix} = \begin{bmatrix}
0 \\
0
\end{bmatrix}\]
\[x_1 = \frac{4x_2}{3}\]
Podemos aplicar o Teorema Espectral na matrix B por ela também ser simétrica, então os autovetores são:
\[V_1 = \langle(4,3)\rangle\]
\[V_3 = \langle(-3,4)\rangle\]

\exercise*
Queremos encontrar a matriz $(T)_\epsilon$. Como é um operador autoadjunto, $a_{12} = a_{21}$, então o operador possui o seguinte formato:
\[(T)_\epsilon = \begin{bmatrix}
a & b\\
b & d
\end{bmatrix}\]
Como os autovalores são 2 e 3, eles serão a solução do polinômio característico, que é calculado a partir do determinante da matrix:
\[det\left(\begin{bmatrix}
a-t & b\\
b & d-t
\end{bmatrix}\right) = (t-2)(t-3)\]
\[(a-t)(d-t)-b^2 = (t-2)(t-3)\]
\[t^2 + (-a-d)t + (ad-b^2) = t^2 -5t +6\]
\[\begin{cases}
a+d=5\\
ad-b^2=6
\end{cases}\]
No entanto, temos 3 variáveis e 2 equações, precisa de mais uma. Para isso, podemos usar o fato que o vetor $(3,4)$ é o autovetor associado ao autovalor 2.
\[\begin{bmatrix}
a-2 & b\\
b & d-2
\end{bmatrix}
\begin{bmatrix}
3\\
4
\end{bmatrix} = 0\]
\[\begin{cases}
3a + 4b = 6\\
3b + 4d = 8
\end{cases}\]
Disso tiramos que o operador $(T)_\epsilon$ é:
\[\frac{1}{25}\begin{bmatrix}
66 & -12\\
-12 & 59
\end{bmatrix}\]

\exercise*
Escolhendo 2 vetores ortogonais entre si, é possível fazer um triângulo retângulo. Como o operador é autoadjunto, há exatamente 2 autovetores e que são ortogonais entre si, então vamos encontrar esses autovetores e transformá-los em um triângulo isósceles, para manter o mesmo ângulo nas transformações:
\[det\left(\frac{1}{13}\begin{bmatrix}
-23-t & 24\\
24 & -3-t
\end{bmatrix}\right) = (t+\frac{23}{13})(t+\frac{3}{13})-\frac{576}{169} = t^2 + 2t - 3 = (t+3)(t-1)\]
Então os autovalores são -3 e 1. Para encontrar o autovetor relacionado ao autovalor 1:
\[\frac{1}{13}\begin{bmatrix}
-36x_1 + 24x_2\\
24x_1 -16x_2
\end{bmatrix}\ = 0\]
\[x_2 = \frac{3}{2}x_1\]
Então $V_1 = \langle(2,3)\rangle$. Por ser autoadjunto, $V_{-3} = \langle(-3,2)\rangle$. Normalizando esses vetores:
\[V_1 = \frac{1}{\sqrt{13}}\langle(2,3)\rangle\]
\[V_{-3} = \frac{1}{\sqrt{13}}\langle(-3,2)\rangle\]

Como o triângulo isósceles também é retângulo, os catetos serão iguais. Chamando os catetos de $a$ e a hipotenusa de $b$, temos as seguintes propriedades:
\[2a^2 = b^2 \Rightarrow b = \sqrt{2}a\]
\[2a + b = 9 \Rightarrow 2a + \sqrt{2}a = 9 \Rightarrow a = \frac{9}{2+\sqrt{2}} = \frac{9(\sqrt{2}-1)}{\sqrt{2}}\]


Queremos então que os autovetores transformados tenham esse módulo, como o vetor original é multiplicado pelo autovalor, basta dividir esse módulo pelos autovalores. Sejam $u_1$ e $u_2$ os vetores que formam o triângulo retângulo.

\[u_1 = \frac{9(\sqrt{2}-1)}{\sqrt{2}}\frac{1}{\sqrt{13}}(2,3) = \frac{9(\sqrt{2}-1)}{\sqrt{26}}(2,3)\]
\[u_2 = \frac{9(\sqrt{2}-1)}{\sqrt{2}}\frac{1}{-3}\frac{1}{\sqrt{13}}(-3,2) = \frac{-3(\sqrt{2}-1)}{\sqrt{26}}(-3,2)\]



\exercise*
Como -1 é o autovalor associado ao autovetor (2,-1), podemos usar esse fato para encontrar os valores de $a$ e $b$.
\[\begin{bmatrix}
0-(-1) & b\\
2 & a-(-1)
\end{bmatrix}
\begin{bmatrix}
2\\
-1
\end{bmatrix} = 0\]
Igualando, obtém-se o seguinte sistema:
\[\begin{cases}
2-b=0\\
4-(a+1)=0
\end{cases}\]
\[b=2\]
\[a=3\]
Então a matriz é:
\[\begin{bmatrix}
0 & 2\\
2 & 3
\end{bmatrix}\]

\exercise
(a) 
\[\left[\begin{array}{cccc|c}
1 & -3 & 3 & -3 & 4\\
2 & -5 & 2 & -5 & 8\\
1 & -3 & 6 & -5 & 1\\
-3 & 6 & 3 & 2 & 14
\end{array}\right] \xrightarrow{L_{1,2}(-2)}
\left[\begin{array}{cccc|c}
1 & -3 & 3 & -3 & 4\\
0 & 1 & -4 & 1 & 0\\
1 & -3 & 6 & -5 & 1\\
-3 & 6 & 3 & 2 & 14
\end{array}\right] \xrightarrow{L_{1,3}(-1)}\]
\[\left[\begin{array}{cccc|c}
1 & -3 & 3 & -3 & 4\\
0 & 1 & -4 & 1 & 0\\
0 & 0 & 3 & -2 & -3\\
-3 & 6 & 3 & 2 & 14
\end{array}\right] \xrightarrow{L_{1,4}(3)}
\left[\begin{array}{cccc|c}
1 & -3 & 3 & -3 & 4\\
0 & 1 & -4 & 1 & 0\\
0 & 0 & 3 & -2 & -3\\
0 & -3 & 12 & -7 & 26
\end{array}\right] \xrightarrow{L_{2,4}(3)}
\]
\[\left[\begin{array}{cccc|c}
1 & -3 & 3 & -3 & 4\\
0 & 1 & -4 & 1 & 0\\
0 & 0 & 3 & -2 & -3\\
0 & 0 & 0 & -4 & 26
\end{array}\right]
\]
Substituindo de baixo pra cima obtemos:
\[x_4 = \frac{-13}{2}\]
\[x_3 = \frac{-16}{3}\]
\[x_2 = \frac{-89}{6}\]
\[x_1 = -44\]


(b)
\[\left[\begin{array}{cccc|c}
1 & -3 & -4 & -1 & -4\\
2 & -4 & -6 & -4 & -4\\
4 & -8 & -13 & -8 & -5\\
-14 & 28 & 45 & 28 & 10
\end{array}\right] \xrightarrow{L_{1,2}(-2)}
\left[\begin{array}{cccc|c}
1 & -3 & -4 & -1 & -4\\
0 & 2 & 2 & -2 & 4\\
4 & -8 & -13 & -8 & -5\\
-14 & 28 & 45 & 28 & 10
\end{array}\right] \xrightarrow{L_{1,3}(-4)}\]
\[\left[\begin{array}{cccc|c}
1 & -3 & -4 & -1 & -4\\
0 & 2 & 2 & -2 & 4\\
0 & 4 & 3 & -4 & 11\\
-14 & 28 & 45 & 28 & 10
\end{array}\right] \xrightarrow{L_{1,4}(14)}
\left[\begin{array}{cccc|c}
1 & -3 & -4 & -1 & -4\\
0 & 2 & 2 & -2 & 4\\
0 & 4 & 3 & -4 & 11\\
0 & -14 & -11 & 14 & -46
\end{array}\right] \xrightarrow{L_{2,3}(-2)}\]
\[\left[\begin{array}{cccc|c}
1 & -3 & -4 & -1 & -4\\
0 & 2 & 2 & -2 & 4\\
0 & 0 & -1 & 0 & 3\\
0 & -14 & -11 & 14 & -46
\end{array}\right] \xrightarrow{L_{2,4}(7)}
\left[\begin{array}{cccc|c}
1 & -3 & -4 & -1 & -4\\
0 & 2 & 2 & -2 & 4\\
0 & 0 & -1 & 0 & 3\\
0 & 0 & 3 & 0 & -18
\end{array}\right]\]
As duas últimas equações entram em conflito, logo, não há soluções.

\exercise
(a) Transformando em uma matriz escada:

\[\left[\begin{array}{cccc|c}
1 & -1 & 0 & 1 & 2\\
2 & -3 & 1 & 3 & 4\\
-1 & 5 & k-4 & -4 & -3\\
-4 & 8 & 3k-4 & k^2-k-5 & k-12
\end{array}\right] \xrightarrow{L_{1,2}(-2)}
\left[\begin{array}{cccc|c}
1 & -1 & 0 & 1 & 2\\
0 & -1 & 1 & 1 & 0\\
-1 & 5 & k-4 & -4 & -3\\
-4 & 8 & 3k-4 & k^2-k-5 & k-12
\end{array}\right] \xrightarrow{L_{1,3}(1)}\]
\[\left[\begin{array}{cccc|c}
1 & -1 & 0 & 1 & 2\\
0 & -1 & 1 & 1 & 0\\
0 & 4 & k-4 & -3 & -1\\
-4 & 8 & 3k-4 & k^2-k-5 & k-12
\end{array}\right] \xrightarrow{L_{1,4}(4)}
\left[\begin{array}{cccc|c}
1 & -1 & 0 & 1 & 2\\
0 & -1 & 1 & 1 & 0\\
0 & 4 & k-4 & -3 & -1\\
0 & 4 & 3k-4 & k^2-k-1 & k-4
\end{array}\right] \xrightarrow{L_{2,3}(4)}\]
\[\left[\begin{array}{cccc|c}
1 & -1 & 0 & 1 & 2\\
0 & -1 & 1 & 1 & 0\\
0 & 0 & k & 1 & -1\\
0 & 4 & 3k-4 & k^2-k-1 & k-4
\end{array}\right] \xrightarrow{L_{2,4}(4)}
\left[\begin{array}{cccc|c}
1 & -1 & 0 & 1 & 2\\
0 & -1 & 1 & 1 & 0\\
0 & 0 & k & 1 & -1\\
0 & 0 & 3k & k^2-k+3 & k-4
\end{array}\right] \xrightarrow{L_{3,4}(-3)}\]
\[\left[\begin{array}{cccc|c}
1 & -1 & 0 & 1 & 2\\
0 & -1 & 1 & 1 & 0\\
0 & 0 & k & 1 & -1\\
0 & 0 & 0 & k^2-k & k-1
\end{array}\right]\]
Se $k=0$ então $k^2-k=0$. Se isso acontecer, na última linha diz que a soma de todas as variáveis com coeficiente 0 dá o resultado -1, o que torna o sistema impossível.

Se $k=1$, a última linha se torna nula, então o sistema fica indeterminado, pois a quantidade de linhas não nulas fica menor que o número de colunas de coeficientes.

Então o sistema é:

\begin{center}
Indeterminado, se $k=1$\\
Impossivel, se $k=0$\\
Determinado, se $k\neq1$ e $k\neq0$
\end{center}

(b)
\[\left[\begin{array}{cccc|c}
1 & -1 & 0 & 1 & 2\\
-3 & 2 & 1 & -2 & -6\\
-3 & 1 & k+2 & 0 & -7\\
4 & -6 & k+2 & k^2-k+7 & k+6
\end{array}\right] \xrightarrow{L_{1,2}(3)}
\left[\begin{array}{cccc|c}
1 & -1 & 0 & 1 & 2\\
0 & -1 & 1 & 1 & 0\\
-3 & 1 & k+2 & 0 & -7\\
4 & -6 & k+2 & k^2-k+7 & k+6
\end{array}\right] \xrightarrow{L_{1,3}(3)}\]
\[\left[\begin{array}{cccc|c}
1 & -1 & 0 & 1 & 2\\
0 & -1 & 1 & 1 & 0\\
0 & -2 & k+2 & 3 & -1\\
4 & -6 & k+2 & k^2-k+7 & k+6
\end{array}\right] \xrightarrow{L_{1,4}(-4)}
\left[\begin{array}{cccc|c}
1 & -1 & 0 & 1 & 2\\
0 & -1 & 1 & 1 & 0\\
0 & -2 & k+2 & 3 & -1\\
0 & -2 & k+2 & k^2-k+3 & k-2
\end{array}\right] \xrightarrow{L_{2,3}(-2)}\]
\[\left[\begin{array}{cccc|c}
1 & -1 & 0 & 1 & 2\\
0 & -1 & 1 & 1 & 0\\
0 & 0 & k & 1 & -1\\
0 & -2 & k+2 & k^2-k+3 & k-2
\end{array}\right] \xrightarrow{L_{2,4}(-2)}
\left[\begin{array}{cccc|c}
1 & -1 & 0 & 1 & 2\\
0 & -1 & 1 & 1 & 0\\
0 & 0 & k & 1 & -1\\
0 & 0 & k & k^2-k+1 & k-2
\end{array}\right] \xrightarrow{L_{3,4}(-1)}\]
\[\left[\begin{array}{cccc|c}
1 & -1 & 0 & 1 & 2\\
0 & -1 & 1 & 1 & 0\\
0 & 0 & k & 1 & -1\\
0 & 0 & 0 & k^2-k & k-1
\end{array}\right]\]
Como essa matriz escada foi igual à da letra A, então a resposta é a mesma:

\begin{center}
Indeterminado, se $k=1$\\
Impossivel, se $k=0$\\
Determinado, se $k\neq1$ e $k\neq0$
\end{center}

\end{document}
